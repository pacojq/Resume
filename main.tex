%%%%%%%%%%%%%%%%%%%%%%%%%%%%%%%%%%%%%%%%%
% Cies Resume/CV
% LaTeX Template
% Version 1.1 (20/7/14)
%
% This template has been downloaded from:
% http://www.LaTeXTemplates.com
%
% Original author:
% Cies Breijs (cies@kde.nl)
% https://github.com/cies/resume with extensive modifications by:
% Vel (vel@latextemplates.com)
%
% License:
% CC BY-NC-SA 3.0 (http://creativecommons.org/licenses/by-nc-sa/3.0/)
%
%%%%%%%%%%%%%%%%%%%%%%%%%%%%%%%%%%%%%%%%%

%----------------------------------------------------------------------------------------
%	PACKAGES AND OTHER DOCUMENT CONFIGURATIONS
%----------------------------------------------------------------------------------------

\documentclass[10pt,a4paper]{article} % Font size (10-12pt) and paper size (a4paper, letterpaper, legalpaper, etc)

% Copyright (c) 2012 Cies Breijs
%
% The MIT License
%
% Permission is hereby granted, free of charge, to any person obtaining a copy
% of this software and associated documentation files (the "Software"), to deal
% in the Software without restriction, including without limitation the rights
% to use, copy, modify, merge, publish, distribute, sublicense, and/or sell
% copies of the Software, and to permit persons to whom the Software is
% furnished to do so, subject to the following conditions:
%
% The above copyright notice and this permission notice shall be included in
% all copies or substantial portions of the Software.
%
% THE SOFTWARE IS PROVIDED "AS IS", WITHOUT WARRANTY OF ANY KIND, EXPRESS OR
% IMPLIED, INCLUDING BUT NOT LIMITED TO THE WARRANTIES OF MERCHANTABILITY,
% FITNESS FOR A PARTICULAR PURPOSE AND NONINFRINGEMENT. IN NO EVENT SHALL THE
% AUTHORS OR COPYRIGHT HOLDERS BE LIABLE FOR ANY CLAIM, DAMAGES OR OTHER
% LIABILITY, WHETHER IN AN ACTION OF CONTRACT, TORT OR OTHERWISE, ARISING FROM,
% OUT OF OR IN CONNECTION WITH THE SOFTWARE OR THE USE OR OTHER DEALINGS IN THE
% SOFTWARE.

%%% LOAD AND SETUP PACKAGES

\usepackage[margin=0.75in]{geometry} % Adjusts the margins

\usepackage{multicol} % Required for multiple columns of text

\usepackage{mdwlist} % Required to fine tune lists with a inline headings and indented content

\usepackage{relsize} % Required for the \textscale command for custom small caps text

\usepackage{hyperref} % Required for customizing links
\usepackage{xcolor} % Required for specifying custom colors
\definecolor{dark-blue}{rgb}{0.15,0.15,0.4} % Defines the dark blue color used for links
\hypersetup{colorlinks,linkcolor={dark-blue},citecolor={dark-blue},urlcolor={dark-blue}} % Assigns the dark blue color to all links in the template

\usepackage{tgpagella} % Use the TeX Gyre Pagella font throughout the document
\usepackage[T1]{fontenc}
\usepackage{microtype} % Slightly tweaks character and word spacings for better typography

\pagestyle{empty} % Stop page numbering

%----------------------------------------------------------------------------------------
%	DEFINE STRUCTURAL COMMANDS
%----------------------------------------------------------------------------------------

\newenvironment{indentsection} % Defines the indentsection environment which indents text in sections titles
{\begin{list}{}{\setlength{\leftmargin}{\newparindent}\setlength{\parsep}{0pt}\setlength{\parskip}{0pt}\setlength{\itemsep}{0pt}\setlength{\topsep}{0pt}}}{\end{list}}

\newcommand*\maintitle[2]{\noindent{\LARGE \textbf{#1}}\ \ \ \emph{#2}\vspace{0.3em}} % Main title (name) with date of birth or subtitle

\newcommand*\roottitle[1]{\subsection*{#1}\vspace{-0.3em}\nopagebreak[4]} % Top level sections in the template

\newcommand{\headedsection}[3]{\nopagebreak[4]\begin{indentsection}\item[]\textscale{1.1}{#1}\hfill#2#3\end{indentsection}\nopagebreak[4]} % Section title used for a new employer

\newcommand{\headedsubsection}[3]{\nopagebreak[4]\begin{indentsection}\item[]\textbf{#1}\hfill\emph{#2}#3\end{indentsection}\nopagebreak[4]} % Section title used for a new position

\newcommand{\bodytext}[1]{\nopagebreak[4]\begin{indentsection}\item[]#1\end{indentsection}\pagebreak[2]} % Body text (indented)

\newcommand{\inlineheadsection}[2]{\begin{basedescript}{\setlength{\leftmargin}{\doubleparindent}}\item[\hspace{\newparindent}\textbf{#1}]#2\end{basedescript}\vspace{-1.7em}} % Section title where body text starts immediately after the title

\newcommand*\acr[1]{\textscale{.85}{#1}} % Custom acronyms command

\newcommand*\bull{\ \ \raisebox{-0.365em}[-1em][-1em]{\textscale{4}{$\cdot$}} \ } % Custom bullet point for separating content

\newlength{\newparindent} % It seems not to work when simply using \parindent...
\addtolength{\newparindent}{\parindent}

\newlength{\doubleparindent} % A double \parindent...
\addtolength{\doubleparindent}{\parindent}

\newcommand{\breakvspace}[1]{\pagebreak[2]\vspace{#1}\pagebreak[2]} % A custom vspace command with custom before and after spacing lengths
\newcommand{\nobreakvspace}[1]{\nopagebreak[4]\vspace{#1}\nopagebreak[4]} % A custom vspace command with custom before and after spacing lengths that do not break the page

\newcommand{\spacedhrule}[2]{\breakvspace{#1}\hrule\nobreakvspace{#2}} % Defines a horizontal line with some vertical space before and after it % Include structure.tex which contains packages and document layout definitions

\renewcommand{\thefootnote}{\roman{footnote}}

\hyphenation{Some-long-word} % Specify custom hyphenation points in words with dashes where you would like hyphenation to occur, or alternatively, don't put any dashes in a word to stop hyphenation altogether

\begin{document} 

%----------------------------------------------------------------------------------------
%	NAME AND CONTACT INFORMATION
%----------------------------------------------------------------------------------------

\maintitle{Paco Juan Quirós}{TOOLS PROGRAMMER RESUME}

\noindent\href{mailto:pacojuanquiros@gmail.com}{pacojuanquiros@gmail.com}\bull % Your email address
\textsmaller{+}34 660 97 50 09\bull \href{http://www.github.com/pacojq}{pacojq} \textit{(GitHub)}\bull % Your phone number(s) and Skype username
\href{http://pacojq.github.io}{pacojq.github.io} \\ % Your URL
Oviedo, Asturias\bull Spain \bull \textit{As it was in December 2022}% Your address

\spacedhrule{0.9em}{-0.4em} % Horizontal rule - the first bracket is whitespace before and the second is after

%----------------------------------------------------------------------------------------
%	SUMMARY SECTION
%----------------------------------------------------------------------------------------

%\roottitle{Objective} % Root section title

%\vspace{-1.3em} % Reduce whitespace after the Summary heading and the two-column content

%\begin{multicols}{2}  % Start a two-column layout
%\noindent \textit{A summary of your interests, achievements, history, topic of study or any other short summary of your professional life.}\\\\
%Software engineer with experience on performance-critical systems, seeking positions on core game technology teams.
%\end{multicols}

%\spacedhrule{0.5em}{-0.4em} % Horizontal rule - the first bracket is whitespace before and the second is after

%----------------------------------------------------------------------------------------
%	EXPERIENCE SECTION
%----------------------------------------------------------------------------------------

\roottitle{Experience} % Top level section

\headedsection % Employer name which can include a hyperlink and location/URL on the right side of the page
{\href{https://meteorbytestudios.com}{Meteorbyte Studios}}
{\textsc{Oviedo, Asturias, Spain}} {


\headedsubsection % Job title entry for the current employer
{Porting Engineer}
{Aug '21 -- Present}
{\bodytext{
Game Toolset: \textit{Ignited Steel}.
\begin{itemize}
\item{
Implemented a "Sprite Stack" mesh generator tool, which allows the preview and generation of sliced mesh assets from a given set of sprites, to give the game a pixelated 3D look.}
\item{
Extended Unity's Editor with a Localization Scanner, with the ability of checking the whole asset database looking for unlocalised UI elements or miss-configured font asset references.}
\end{itemize}
}}


\headedsubsection % Job title entry for the current employer
{Senior Programmer}
{Aug '19 -- Present}
{
\bodytext{
Tool: \textit{Rosetta}.
\begin{itemize}
\item{
Proprietary localization tool written in C++, using XML, OpenGL and ImGui. Used for the game localization process in Meteorbyte Studios releases. Provides basic functionality, such as basic project statistics, text search and filtering, content preview with different fonts and character sets, and string comparison among different languages.}
\end{itemize}
}
\bodytext{
Game Toolset: \textit{Unannounced Game}.
\begin{itemize}
\item{
Developed a node-based editor for Narrative Designers to structure both quests and dialogue assets in-engine.}
\item{
Extended Unity with a custom level editor tool for procedural asset placement.}
\end{itemize}
}
\bodytext{
Game Toolset: \textit{Deck RX}.
\begin{itemize}
\item{
Built a tile-based level editor for in-game circuits. Programmed in C\# and outputting level data in both JSON and binary formats.}
\end{itemize}
}
\bodytext{
Game Toolset: \textit{Attrah}\footnote{Project selected by the Spanish Government's Ministry of Culture as one of the best cultural projects of 2019.}.
\begin{itemize}
\item{
Used \href{https://buildbot.net}{Buildbot} to develop an internal Continuous Integration tool, which would automate the build process for PC and PS4 after major changes in Version Control.}
\item{
Developed a C\# library for visual profiling which allows spotting CPU bottlenecks with no extra cost on performance. Data visualization using Google Chrome's \textit{Tracing} tool. Code on \href{https://github.com/pacojq/ChromeTracing.NET}{GitHub}.}
\end{itemize}
}
}

\headedsubsection % Job title entry for the current employer
{Junior Programmer}
{Aug '18 -- Aug '19}
{\bodytext{
Game: \href{https://store.steampowered.com/app/878420/Woodpunk/}{\textit{Woodpunk}} - 2018.
\begin{itemize}
%\item{
%Refactored the already-existing platform integration module -highly coupled to the Steam API- to work in a multi-platform environment (Play Station 4 and Xbox One)\footnote{Release of ports to PS4 and Xbox One on hold for administrative reasons.}.}
\item{
Developed a command line tool to generate Steam localized Store Page and achievement content from in-game localization files.}
\end{itemize}
}}

}

%------------------------------------------------

%\headedsection % Employer name which can include a hyperlink and location/URL on the right side of the page
%{\href{https://bipolardawn.github.io}{Bipolar Dawn}}
%{\textsc{Avilés, Asturias, Spain}} {

%\headedsubsection % Job title entry for the current employer
%{Owner}
%{Nov '16 -- Present}
%{\bodytext{
%Game: \textit{World Soccer Strikers '91} - 2020-21.
%\begin{itemize}
%\item{
%Extended Unity's render pipeline with a pass that allows recording TV-style replays with secondary cameras. Outputs files with a maximum budget of 8 MB, so they can be shared in Discord.}
%\end{itemize}
%}}

%}

%------------------------------------------------

\begin{center}
\textit{Please refer to \href{https://www.linkedin.com/in/paco-juan-6589ba14b/}{my Linkedin profile} for the complete list of work experiences.}
\end{center}

%------------------------------------------------

\spacedhrule{-0.2em}{-0.4em} % Horizontal rule - the first bracket is whitespace before and the second is after


%----------------------------------------------------------------------------------------
%	PROJECTS SECTION
%----------------------------------------------------------------------------------------

\roottitle{Personal Projects} % Top level section

{

\headedsubsection % Job title entry for the current employer
{\href{https://github.com/pacojq/Xaloc}{Xaloc Engine}}
{Spring 2020 -- Present}
{\bodytext{
In-development 2D engine written in C++, with C\# scripting using mono.
}}

%------------------------------------------------

%\headedsubsection % Job title entry for the current employer
%{\href{https://github.com/pacojq/Seagull}{Seagull}}
%{Spring 2019}
%{\bodytext{
%Toy programming language, compiling to intermediate language. Targeting \textit{MAPL Virtual Machine}, an academic virtual machine built at University of Oviedo.
%}}

}

%------------------------------------------------

\begin{center}
\textit{Please refer to \href{https://github.com/pacojq}{my GitHub account} and my \href{http://pacojq.github.io}{personal webpage} for the complete list of personal projects and research.}
\end{center}

%------------------------------------------------

\spacedhrule{-0.2em}{-0.4em} % Horizontal rule - the first bracket is whitespace before and the second is after


%----------------------------------------------------------------------------------------
%	EDUCATION SECTION
%----------------------------------------------------------------------------------------

\roottitle{Education} % Top level section

\headedsection % Employer name which can include a hyperlink and location/URL on the right side of the page
{\href{http://www.uniovi.es/en/inicio}{University of Oviedo}}
{\textsc{Oviedo, Spain}} {

\headedsubsection % Job title entry for the current employer
{Bachelor degree in Software Engineering}
{Sep '16 -- Feb '21}
{\bodytext{
Bilingual degree: 50\% English / 50\% Spanish. Proposed to graduate with Honors.
}}

}

\spacedhrule{0.5em}{-0.4em} % Horizontal rule - the first bracket is whitespace before and the second is after

%----------------------------------------------------------------------------------------
%	SKILLS SECTION
%----------------------------------------------------------------------------------------

\roottitle{Skills} % Top level section

\inlineheadsection % Special section that has an inline header with a 'hanging' paragraph
{Languages:}
{C\#, C++, Java, C, HLSL/GLSL, Python, Scheme.}

%------------------------------------------------

\inlineheadsection % Special section that has an inline header with a 'hanging' paragraph
{Game development:}
{PlayStation 4 \& 5, Xbox One \& Series X|S, Nintendo Switch, OpenGL, Unity, Mono.}

%------------------------------------------------

\inlineheadsection % Special section that has an inline header with a 'hanging' paragraph
{Other tools:}
{Visual Studio, Eclipse, JetBrains Rider, RenderDoc, Buildbot, Git, ANTLR, Amazon AWS.}

%------------------------------------------------

\inlineheadsection % Special section that has an inline header with a 'hanging' paragraph
{Natural languages:}
{Catalan \textit{(native)}, Spanish \textit{(native)}, English \textit{(bilingual)} and French \textit{(elementary)}.}

%------------------------------------------------

\end{document}